\newcommand{\figurescaling}{0.7}
\documentclass[a4paper,oneside,parskip=half]{scrartcl}

\usepackage{graphicx}
\usepackage{natbib}
\usepackage[utf8]{inputenc}
\usepackage{tabularx}
\usepackage{hyperref}
\usepackage{color}
\usepackage[usenames,dvipsnames,svgnames,table]{xcolor}
\usepackage{longtable}
\usepackage{slashbox}
\usepackage{multirow}
\usepackage{minibox}
\usepackage{placeins }
\usepackage{booktabs}
\usepackage[UKenglish]{babel}
\usepackage[T1]{fontenc}
\usepackage{lmodern}
\usepackage{pifont}
\usepackage{paralist}
\usepackage{listings}
\usepackage{enumitem}
\usepackage{multicol}
\usepackage[utf8]{inputenc}
\usepackage{authblk}

\title{Machine Learning WS2013}
\subtitle{Assignment 3 \\[0.8cm] {\rmfamily\normalfont\Large}}

\author[1]{Dino Rossegger}
\author[2]{Navid Rekabsaz}
\author[3]{Soroosh Mortezapoor}
\affil[1]{e0926471@student.tuwien.ac.at}
\affil[2]{e1129057@student.tuwien.ac.at}
\affil[3]{soroosh.mortezapoor@student.tuwien.ac.at}

\lstset{language=java,basicstyle=\ttfamily\footnotesize}

\begin{document}

\maketitle
\begin{abstract}
\end{abstract}

\section{Task}
The task was to analyse the differences in different feature selection techniques, i.e. analyse the difference in the selected features of the techniques. Therefore a java program had to be developed using the weka library to run the feature selection techniques on different datasets supplied by the lectors.

Furthermore a comparison of the results of k-nn using the top $n$ features of each feature selection technique and using mutual features was to be done. This was also implemented in the program using weka.

\section{Program}
The java program uses ant to allow easy building and running of the program. In the build file the two properties \lstinline$weka.dir$ and \lstinline$lib.dir$ (containing apache commons cli) need to be changed.
Afterwards the program can be compiled using \lstinline$ant compile$, it is suggested to create a jar file using \lstinline$ant jar$.

The program accepts a number of arguments, in Listing~\ref{list:helpoutput} the help output can be seen. The only obligatory flag is \lstinline$-d <directory>$ specifying the directory from which the instances should be read. The rest of the arguments are optional, the help output in Listing~\ref{list:helpoutput} should be enough aid in understanding the usage of the program.
\begin{lstlisting}[caption=help output of the program, label=list:helpoutput]
 -c               compare results and find mutual features of result files
                   inside a directory
 -d <directory>   directory with instancefiles in csv or arff format
 -f <technique>   use specified feature selection technique
 -h               display this usage information
 -l               list all available feature selection techniques
 -n <n>           use top <n> attributes, default: 10
 -t <f>           consider attributes appearing in f per cent of the
                   result sets, default: 0.5

\end{lstlisting}
First the specified techniques (or all implemented techniques if \lstinline$-f$ is not set) are applied to the datasets, then the selected attributes are compared. At last a classifier is run using the features obtained from the feature selection methods. The precision, recall and f1-scores of the runs are printed on standard output.

\section{Technique Comparison}
\section{Comparison on Classifier}
\end{document}
