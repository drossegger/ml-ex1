\section{First Order Theorem Proving}
\label{db:sec:ds2}
\subsection{Description}
This database is made of $6112$ instances made of $51$ attributes each. This $51$ attributes are static and dynamic features of first order theorems which were tried to be solved with $5$ different heuristics. The last $5$ columns cotain the runtime of the heuristics or $-100$ ifthe heuristic was not able to prove the theorem within $100$ seconds.\par
Our first idea was to assign a class $1-5$ to each instance indicating which heuristic was the fastest or $0$ if the theorem could not be proved by any heuristic within $100$ seconds. Running experiments with this configuration we observed that all our machine learning algorithms did not produce any good results. While the hit rate for instances which were unsolvable was quite good, it seemed impossible to predict which heuristic was the fastest in most cases. Therefore we could not achieve a precision higher than $60\%$.
Looking at the instances the cause of this became obvious. For most of the provable algorithms the difference in runtime of the heuristic was very small, in some cases all five of them had the same runtime. Because of this we changed our configuration, assigning classes to the instances based on the runtime of the best heuristic. The classes can be seen in Table~\ref{ds2:table:classes}.
\begin{table}[h]
	\begin{center}
	\begin{tabular}{c|c|c|c|c|c|c}

		Label & $1$ & $2$ & $3$ & $4$ & $5$& $0$\\\hline
		Runtime (s) & $<1$ & $<10$ & $<25$ & $<50$ &$<100$ &$>100$\\
	\end{tabular}
\end{center}
	\caption{Class assignment \label{ds2:table:classes}}
\end{table}
\subsection{Preprocessing}
We applied min max scaling as well as mean removal and variance scaling. It turned out that mean removal and variance scaling was the best scling method. Imputation of missing values was not needed for this dataset but two attributes were removed because of redundancy.
\subsection{Logistic Regression}

\subsection{Decision Tree}

\subsection{Support Vector Machine}

\subsection{Neural Networks}

\subsection{Comparison}
