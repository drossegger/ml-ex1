\section{Funniest Youtube Video}
\label{db:sec:ds3}
\subsection{Description}

The third dataset used during this experiment is called funniest youtube video.
The dataset utilized for classification is called YouTube Comedy Slam Preference
Data Data Set~\cite{}. The subject of this data is fairly interesting and
challenging in many aspects. The dataset contains over $1,200,000$ number
of instances. Each instance comprises two strings and a label where strings
refer to IDs of videos on Youtube and the label expresses a vote by a
user that mentions which one was funnier in the voter's opinion, either 'left'
or 'right'.

\subsection{Preparing data}

As mentioned in latest section, the dataset contains only ID of a video and no
more details. Thus it is up to the experimenter to select and find some features
for each video. Luckily, Youtube provides a robust, easy-to-use API ~\cite{}
which offers many useful data and information about videos and they are accessible having
only ID of a video and a bit of knowledge about web technologies and
exchange data structures. 

Nonetheless there are some obstacles and defficulties
in this path. The very first one is that many of the videos provided in the
dataset are deleted either by uploders or Youtube due to copyright violations.
Second problem is that Youtube consider immidiate consequtive requests from a
client as a fraud and cuts information flow. As a result, the pace at which data
acquisition is happening falls steeply due to many halts in the process of
data acquisition to prevent blocking of data flow. This is why at the end, only
$500,000$ instances of the dataset were inspected in a four-day non-stop data
acquisition process, a process in which data on only around $50\%$ of the
inspected instances, both right and left videos were found. Consequently, the
final data set on which learning is performed, totally contains around $250,000$
instances. Later we will see that this set is splitted to three different sets
for further use.

As it is mentioned, Youtube API provides service clients with many diverse types
of information about videos from which some important features were selected for
the final dataset, raw-dataset hereafter. In raw-dataset there is an instance
row number, two IDs for two videos, a label indicating either 'left' or 'right'
video and some new features. These new features are distinguished with 
$R_{\ldots}$ and $L_{\ldots}$ for the right video and the left video
respectively.
Features of the raw-dataset are shown in Table~\ref{ds3:table:attributes}.

Surprisingly, there are some missing values in the responses by Youtube API in
the requested information of some videos. Therefore we have to deal with these
missing values before reaching a reasonalbe model which is capable of predicting
new entries.


\begin{table}[p]
	\begin{center}
	\begin{tabular}{||l|l||}
		\hline
		
		{\bf  Name} & {\bf Description}\\ \hline
		Index & Row number.\\ \hline
		LID & Youtube Id of the left video.\\ \hline
		RID & Youtube Id of the right video.\\ \hline
		(+) R\_userId & Uploader of right video.\\ \hline
		(+) R\_commentHits & Number of comments on the right video. \\ \hline
		(+) R\_ratingAverage & Rating average for the right video. $\in [0,5]$\\
		\hline (+) R\_ratingMin & Mininum of rating for the right video. $\in
		\{0,..,5\}$\\
		\hline (+) R\_ratingMax & Maximum of the rating the right video. $\in
		\{0,..,5\}$ \\ \hline
		(+) R\_ratingCount & Number of ratings for the right video.\\ \hline
		(+) R\_duration & Duration of the right video in seconds.\\ \hline
		(+) R\_published & Date at which the right video is uploaded.\\ \hline
		(+) R\_likeCount & Number of likes for the right video.\\ \hline
		(+) R\_dislikeCount & Number of dislikes for the right video.\\ \hline
		(+) R\_viewCount & Number of views for the right video.\\ \hline
		(+) L\_userId & Uploader of left video.\\ \hline
		(+) L\_commentHits & Number of comments on the left video.\\ \hline
		(+) L\_ratingAverage & Rating average for the left video. $\in [0,5]$\\ \hline
		(+) L\_ratingMin & Mininum of rating for the left video. $\in
		\{0,..,5\}$\\ \hline
		(+) L\_ratingMax & Maximum of the rating the left video. $\in
		\{0,..,5\}$\\ \hline
		(+) L\_ratingCount & Number of ratings for the left video.\\ \hline
		(+) L\_duration & Duration of the left video in seconds.\\ \hline
		(+) L\_published & Date at which the left video is uploaded.\\ \hline
		(+) L\_likeCount & Number of likes for the left video.\\ \hline
		(+) L\_dislikeCount & Number of dislikes for the left video.\\ \hline
		(+) L\_viewCount & Number of views for the left video.\\ \hline
		VotedIndex & Label $\in \{'right', 'left'\}$\\ \hline

	\end{tabular}
\end{center}
	\caption{Raw-dataset Attributes. (+) means this attribute is fetched from
	Youtube API.
	\label{ds3:table:attributes}}
\end{table}






\subsection{Preprocessing}

Before we can proceed to build a model for classification, there are some steps
to follow in order to prepare data. These steps include mapping all
non-numerical attributes to numericals due to the fact that most of the
algorithms in classification specially implemented by SciKit only accept
numerical data as input for learning.

\subsubsection{Mapping Date and Time}

First step is to map date and time information to numerical correspondence. In
this experiment, all date and time information are mapped to float numbers by
Epoch~\cite{} approach which what counts the number of seconds since January 
$1^{st}$, $1970$. For instance, {\it 12/16/2013 10:40:21 AM GMT+1} becomes
$1387186821.0$ by this mapping.

\subsubsection{Categorical strings mapping}

Second step is to map all string information to numerical. Better to say, all
categorical strings which exists in the data should be mapped to a new number.
For instance, {\it 'left'} $\to 0$ and {\it 'right'} $\to 1$ in the label
column. This step of preprocessing works on label column as well as
{\it R\_userId} and {\it L\_userId}.

\subsubsection{Encoding categorical features} 

In spite of mapping categorical features to numbers in the previous step, still
a further effort is required before using the data in classifiers. The problem
is that these numbers are not ready for the classifier since the classifier
would consider and interpret them as being ordered while it is not desirable.
One possibility is to convert them to a new form having not only numerical
value, but also solving this problem. 

Luckily there are some facilities in
SciKit such as one-of-K and one-hot encodings, implemented in OneHotEncoder. 
This function transforms each categorical
   feature with $m$ possible values into $m$ binary features, with only one
   active.

Although it looks great and helpful, we will see that it is not always the best
way since in this dataset, the mapping turns the dataset from under $30$
dimensions to over $2700$ dimensions and it is not desirable in many algorithms
which are highly affected by number of attributes such as {\it Support Vector
Machines}.

According to deep increase in the dimensionality of data due to encoding
categorical features, meaning {\it R\_userId} and {\it L\_userId} here in our
dataset, an idea bounces around to not only avoid this encoding, but also to
discard these two features from the dataset. Thus as a solution to control
dimensionality and keep simplicity, this approach is also taken into
consideration.

\subsubsection{Normalisation/Scaling}

It is sometimes crucial to normalise data in order to achieve best result. It is
highly probable that in many algorithms which demand on the distance of data, a
column with larger numbers dominate all other ones. For instance, here we have
seen that $1387186821.0$ as a mapped date and time is considerably lager than a
float value $v_n \in [0,5]$.

Standard scaling and {\it MinMax} scaling are used over the dataset and all
results are shown with and without normalisation later in this report.

\subsubsection{Missing values}

As it is mentioned before, there are some missing values in the dataset. Missing
value is not acceptable by any classifier so there are some approaches to deal
with missing values. These approaches include eliminating instances containing
missing values, using the most frequent value in the column, using the mean of
the column and using the median of the column as a substitution to a missing
value. All methods are tested and result are presented later.

It should be noted that, here in this report all results hereafter are based on
{\it most frequent} method in dealing with missing values, no
normalisation/scaling and no {\it R\_userId} and {\it L\_userId} as the basis of
experiment unless it is restrictly stated that any other data is used.


\subsubsection{Splitting the dataset}

As two previous datasets in sections ~\ref{} and ~\ref{}, current dataset is
also splitted into three diverse subsets with approximately $60\%$, $20\%$ and
respectively $20\%$ of the data as train, test and validation sets.

\subsection{Logistic Regression}

First algorithm by which a model for classification is tried to build is {\it
Logistic Regression}. Not only once, but it is tried four times to find the best
model created by this algorithm using different {\it C} parameter~\ref{} where
$C \in \{0.1,0.5,1,10 \}$. Results can be seen in
Table~\ref{table:ds3:logisticregression}. Also detailed information about the best practice with a combination of number of exponents
and parameter C on the test set is shown in
Table~\ref{table:ds3:logisticregression-test}.

\begin{table}[p]
\begin{center}
\begin{tabular}{|c|c|c|c|c|c|}
\hline \multicolumn{2}{ |c| }{C} & 0.1 & 2 & 5 & 10 \\

\hline \multicolumn{1}{ |c| }{\multirow{3}{*}{Highest exponent} } & 1 & {\bf
0.38} & 0.38 & 0.38 & 0.38 \\

\cline{2-6} & 5 & 0.35 & 0.35 & 0.35 & 0.35 \\

\cline{2-6} & 10 & 0.37 & 0.37 & 0.37 & 0.37 \\

\hline
\end{tabular}

\caption{Funniest Youtube Video - Logistic Regressions F1-score}
\label{table:ds3:logisticregression}
\end{center}
\end{table}

\begin{table}[p]
\begin{center}
\begin{tabular}{|c|c|c|c|c|}
\hline Class & Precision & Recall & F1-score & Support \\

\hline 0 & 0.52 & 0.98 & 0.68 & 25857\\
\hline 1 & 0.53 & 0.02 & 0.06 & 24066\\
\hline avg / total & 0.52 & 0.52 & 0.38 & 49923\\
\hline
\end{tabular}

\caption{Funniest Youtube Video - Logistic Regressions on Validation dataset}
\label{table:ds3:logisticregression-test}
\end{center}
\end{table}

\subsection{Decision Tree}

Second algorithm to apply to the data is {\it Decision Tree}. As before, there
are two modes of decision trees each one having different criterion $c$ where
$ c \in\{ gini,entropy\}$ used to build a model on current data.

Results are depicted in Table~\ref{table:ds3:decisiontree}. Also again detailed
information of the best model is shown in
Table~\ref{table:db2:decisiontree-test}. It is obvious that chosing between
different criteria, doesn't help in finding any better model.

\begin{table}[p]
\begin{center}
\begin{tabular}{|c|c|}
\hline Criterion & F1-Score \\

\hline Gini & 0.52 \\

\hline Entropy & 0.52 \\

\hline
\end{tabular}

\caption{Funniest Youtube Video - Decision Tree F1-score}
\label{table:ds3:decisiontree}
\end{center}
\end{table}


\begin{table}[p]
\begin{center}
\begin{tabular}{|c|c|c|c|c|}
\hline Class & Precision & Recall & F1-score & Support \\

\hline 0 & 0.50 & 0.54 & 0.52 & 24066\\
\hline 1 & 0.54 & 0.49 & 0.51 & 25857\\
\hline avg / total & 0.52 & 0.52 & 0.52 & 49923\\
\hline
\end{tabular}

\caption{Funniest Youtube Video - Decision Tree on Validation dataset}
\label{table:db2:decisiontree-test}
\end{center}
\end{table}

\subsection{$k$-nearest neighbors}

Another engine for classification used in this experiment is {\it
$k$-nearest neighbors}. From our previous experiment on regression, we observed
that $k$-nearest neighbors can be utilized to build a reasonable model and
predictor. In Table~\ref{table:ds3:knn} it is shown that how successful the
classification experiment was with this algorithm using different number of neighbors. As you
can see, best result is gained with $k=5$ and $k=10$ in $F1-Score$.
Consequently, detailed information about the algorithm results with $k=10$ are
presented in Table~\ref{table:ds3:knn-test}.


\begin{table}[p]
\begin{center}
\begin{tabular}{|c|c|}
\hline Number of Neighbors & F1-Score \\

\hline 2 & 0.47 \\
\hline 5 & {\bf 0.51} \\
\hline 10 & {\bf 0.51} \\

\hline
\end{tabular}

\caption{Funniest Youtube Video - K Nearest Neighbors  F1-score}
\label{table:ds3:knn}
\end{center}
\end{table}

\begin{table}[p]
\begin{center}
\begin{tabular}{|c|c|c|c|c|}
\hline Class & Precision & Recall & F1-score & Support \\

\hline 0 & 0.50 & 0.59 & 0.54 & 24066\\
\hline 1 & 0.54 & 0.44 & 0.49 & 25857\\
\hline avg / total & 0.52 & 0.51 & 0.51 & 49923\\
\hline
\end{tabular}

\caption{Funniest Youtube Video - K Nearest Neighbors on Validation dataset ($k=10$)}
\label{table:ds3:knn-test}
\end{center}
\end{table}



\subsection{Support Vector Machine}

Time required by {\it Support Vector Machines} to create a model is highly
dependant on the size of dataset in terms of dimensionality as well as number of
instances. Hence, the process of building a model by support vector machines for
the current dataset was stopped after three days with no results for a SVM with
RBF kernel. Nevertheless, with linear kernel, it becomes feasible to create a
model using SVM. Results brought by SVM are shown in Table~\ref{}


\begin{table}[p]
\begin{center}
\begin{tabular}{|c|c|c|c|c|c|c|c|c|}
\hline \multicolumn{2}{ |c| }{C} & 0.1 & 1 & 5 & 10 & 20 & 50 & 100 \\

\hline \multicolumn{1}{ |c| }{\multirow{2}{*}{Kernel} } & Linear & 0.35 &
0.38 & 0.38 & 0.38 & 0.38 & 0.38 & 0.38\\

\cline{2-9} & RBF & - & - & - & -  & - & - & -\\

\cline{2-9} & Polynomial & - & - & - & -  & - & - & -\\

\hline
\end{tabular}

\caption{Funniest Youtube Video - SVM F1-score}
\label{table:ds3:svm}
\end{center}
\end{table}




\begin{table}[p]
\begin{center}
\begin{tabular}{|c|c|c|c|c|}
\hline Class & Precision & Recall & F1-score & Support \\

\hline 0 & 0.52 & 0.97 & 0.68 & 25857\\
\hline 1 & 0.54 & 0.03 & 0.07 & 24066\\
\hline avg / total & 0.53 & 0.52 & 0.38 & 49923\\
\hline
\end{tabular}

\caption{Funniest Youtube Video - SVM on Validation dataset}
\label{table:ds3:svm-test}
\end{center}
\end{table}


\subsection{Neural Networks}

Due to some problems and limitations in the base on which SciKit is developed,
meaning NumPy~\cite{} library, Neural Networks algorithm of SciKit returns
a result neither reasonable nor trustable. The result from the algorithm is a
program exception with a $x86$ version of Python installed and a
precision/recall equal to zero, meaning that the sum of true positives and false
positives are equal to zero for one of the
classes with a $x64$ version of Python installed.


\subsection{Different preprocessed data}

Playing with possibilities of preprocessing of data resulted in many different
data as train/test/validation sets. It is surprising to observe that the current
combination of preprocessing operations leads to the best results. 

In no case, there is any improvement in replacing missing values with mean or
median of the column rather than most frequent one. Also normalisation and
scaling do not result in any better model in the current dataset.

\subsection{Comparison}

To sum up the section, Table~\ref{} demonstrates best results by all algorithms
in a glimpse. We should admit that even best algorithm does not create a
magnificent model to predict. Knowing that a function $f$ which assigns one of
two labels to a new instance randomely will leads to Precision $=$ Recall $=$
$0.5$ in no time will cause a total disappointment. Thus there must be a lot of
other things to do in order to find funniest video over Youtube. 

However
obviously among all algorithms, decision tree brought slightly better result.
Despite it is strictly stated by data provider that instances of the original
dataset are shuffled to avoid position bias, in some algorithms such as Logistic
Regression family we observe such a bias where recalls for two classes are
$0.03$ and $0.97$ respectively. The same happens to support vector machines.

Another important point to consider is the amount of time required by support
vector machines which is unbelievably higher than all others. Reference to the
cost of this algorithm, presuming a learned support vector machine with encoded
categorical attributes seems impossible with respect to the size of this report.

\subsection{Future work on Funniest Youtube Video}

An idea floating around is to duplicate training data. As stated before,
features for two videos are in the form of $R_{\ldots}$ and $L_{\ldots}$ for the
right and the left video and respectively a label, either {\it 'right'} or
{\it'left'} indicates the video which is funnier. Supposing that we swap
feature values of $video_{left}$ and $video_{right}$ pairwise and inverse the
label, may cause a hugh difference in the result with respect to the fact that
we are sure about truth of the new instance.

This is one of the tasks remains to be tried for further experiments on the
current dataset in order to find a predicting model for funniest video based on
provided information about each video.

\begin{table}[p]
\begin{center}
\begin{tabular}{|p{5cm}|c|c|c|p{2cm}|p{2cm}|}
\hline Algorithm & Precision & Recall & F1-score & Training Time & Prediction Time \\
\hline Logistic Regression, C=0.1, Exponent=1 & 0.52 & 0.52 & 0.38 & 01.027 &
02.890\\
\hline {\bf Decision Tree, Criterion=Gini} & 0.52 & 0.52 & {\bf 0.52} & 13.321 &
1.807\\
\hline K Nearest Neighbors, k=10 & 0.52 & 0.51 & 0.51 & 00.706 & 20.322 \\
\hline SVM, Kernel=Linear, C=1 & 0.53 & 0.52 & 0.38 &
2165.245 & 201.824\\
\hline Neural Network & - & - & - & - & -\\
\hline
\end{tabular}
\caption{Funniest Youtube Video - Comparison}
\label{table:ds3:comparison}
\end{center}
\end{table}




