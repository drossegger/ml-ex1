\section{Auto MPG dataset}
\subsection{Description}
Auto MPG \footnote{http://scikit-learn.org/} contains 8 attributes of 398 cars  and their corresponding MPG (Miles Per Gallon) value. The dataset has some missing value in forth column (horsepower). Shortage of data as well as existing some missing values makes the dataset distinct and interesting for ML. Furthermore because of short execution time, it provides us the opportunity to apply all developed algorithms as well as imputation techniques.

\subsection{Preprocessing}
In order to standardized data, we used standard scale which first transforms the data to center it by removing the mean value of each feature and then scales it by dividing non-constant features by their standard deviation \cite{scikitstandardization}.

In order to impute missing values, we applied three different techniques containing calculating Mean, Median and Most-Frequent value.

\subsection{Regression}
\subsubsection{Linear Regression}
The results of different Linear Regression algorithms with different imputation techniques are shown in Table \ref{table:db1-linearregression} in Appendix A. Ridge Regression was tested by different values for alpha (0.1, 0.5, 1 and 10) and 0.1 seems to be the best value. The final results were very similar. As it shown all the values are rather the same and very biased. By reducing the number of instances in Train Data the final bias reduced significantly. In order to improve the results, we used Multinomial Regressions in the next step.

\subsubsection{Multinomial Regression}

\subsection{Nearest Neighbor}
\subsection{Support Vector Machine}
\subsection{Statistic Gradient Descent}
\subsection{Comparison}
