\section{Introduction}
In this assignment, we applied many different machine learning techniques on three datasets provided by UCI Machine Learning Repositories \footnote{http://archive.ics.uci.edu/ml/datasets.html}. The datasets and their characteristics are shown in Table \ref{table:datasets}.

In order to run ML algorithms, we create a ML platform using Python libraries. SciKit-Learn \footnote{http://scikit-learn.org/} and PyBrain  \footnote{http://pybrain.org/} are two main libraries used in the platform. SciKit-Learn provides a vast variety of techniques for preprocessing, data imputing and ML algorithms. In order to fill the lack of Neural Network algorithms for supervised learning in SciKit-Learn, we used PyBrain libraries.

Eight different techniques were implemented in the platform as follows:
\begin{itemize}
  \item Linear Ridge Regression
  \item Linear Ridge Regression Cross Validation
  \item Linear Bayesian Ridge Regression
  \item Stochastic Gradient Descent
  \item Decision Tree
  \item K-Nearest Neighbors Regression
  \item Support Vector Machine
  \item Neural Network  
\end{itemize}

\begin{center}
\begin{table}
    \begin{tabular}{ | p{2.5cm} | p{1cm} | p{1cm} | p{1cm} | p{7cm} |}
    \hline
    Name & Instances \# & Attributes \# & Missing Values & Reason for Choosing \\ \hline
    YearPrediction MSD & 515345 & 90 & N/A & In contrast to AutoMPG, the dataset is very big and has many attributes.\\ \hline
    \end{tabular}
    \caption{List of datasets selected from UCI Machine Learning Repositories}
    \label{table:datasets}
    \end{table}
\end{center}
