%%%%%%%%%%%%%%%%%%%%%%%%%%%%%%%%%%%%%%%%%
% University/School Laboratory Report
% LaTeX Template
% Version 3.0 (4/2/13)
%
% This template has been downloaded from:
% http://www.LaTeXTemplates.com
%
% Original author:
% Linux and Unix Users Group at Virginia Tech Wiki 
% (https://vtluug.org/wiki/Example_LaTeX_chem_lab_report)
%
% License:
% CC BY-NC-SA 3.0 (http://creativecommons.org/licenses/by-nc-sa/3.0/)
%
%%%%%%%%%%%%%%%%%%%%%%%%%%%%%%%%%%%%%%%%%

%----------------------------------------------------------------------------------------
%    PACKAGES AND DOCUMENT CONFIGURATIONS
%----------------------------------------------------------------------------------------

\documentclass[10pt]{article}

\usepackage{amsmath,amsfonts,amsthm,amssymb}


\usepackage{fancyhdr}

\usepackage{extramarks}
\usepackage{chngpage}



\usepackage{fullpage}
%\usepackage{fourier}

\usepackage{mathtools}


\usepackage{graphicx} % Required for the inclusion of images


\renewcommand{\labelenumi}{\alph{enumi}.} % Make numbering in the enumerate environment by letter rather than number (e.g. section 6)

%\usepackage{times} % Uncomment to use the Times New Roman font

%----------------------------------------------------------------------------------------
%	DOCUMENT INFORMATION
%----------------------------------------------------------------------------------------

\title{A summery on paper \\ \\ {\bf Foundations of Semantic Web Databases}\\ \\by \\ \\ Claudio Gutierrezuc, Carlos Hurtadouc, Alberto O. Mendelzonut \\ \\ \\ \\} % Title

\author{Soroosh \textsc{Mortezapoor}} % Author name

\date{May 2013} % Date for the report

\begin{document}

\maketitle % Insert the title, author and date

\begin{center}
\begin{tabular}{l r}
Date Performed: & May 2013 \\ % Date the experiment was performed
Course: & Database Theory \\ % Partner names
Semester: & SS 2013 % Instructor/supervisor
\end{tabular}
\end{center}

% If you wish to include an abstract, uncomment the lines below
% \begin{abstract}
% Abstract text
% \end{abstract}

%----------------------------------------------------------------------------------------
%	SECTION 1
%----------------------------------------------------------------------------------------

\section{Objective}



\section{Introduction}


\section{Problem}



\section{RDF}

\subsection{RDF Graph}


\subsection{RDFS Vocabulary}

\subsection{Lean graphs}


\section{Semantic of RDF Graph}


\subsection{Semantics of simple RDF graphs}



\subsection{Semantics of RDF graphs with RDFS Vocabulary}

% If you have more than one objective, uncomment the below:
%\begin{description}
%\item[First Objective] \hfill \\
%Objective 1 text
%\item[Second Objective] \hfill \\
%Objective 2 text
%\end{description}

%\subsection{Definitions}
%\label{definitions}
%\begin{description}
%\item[Stoichiometry]
%The relationship between the relative quantities of substances taking part in a reaction or forming a compound, typically a ratio of whole integers.
%\item[Atomic mass]
%The mass of an atom of a chemical element expressed in atomic mass units. It is approximately equivalent to the number of protons and neutrons in the atom (the mass number) or to the average number allowing for the relative abundances of different isotopes. 
%\end{description} 
 
%----------------------------------------------------------------------------------------
%	SECTION 2
%----------------------------------------------------------------------------------------


%----------------------------------------------------------------------------------------
%	BIBLIOGRAPHY
%----------------------------------------------------------------------------------------

\bibliographystyle{unsrt}

\bibliography{science}

%----------------------------------------------------------------------------------------


\end{document}